%
% Complete documentation on the extended LaTeX markup used for Insight
% documentation is available in ``Documenting Insight'', which is part
% of the standard documentation for Insight.  It may be found online
% at:
%
%     http://www.itk.org/

\documentclass{InsightArticle}

\usepackage[dvips]{graphicx}

%%%%%%%%%%%%%%%%%%%%%%%%%%%%%%%%%%%%%%%%%%%%%%%%%%%%%%%%%%%%%%%%%%
%
%  hyperref should be the last package to be loaded.
%
%%%%%%%%%%%%%%%%%%%%%%%%%%%%%%%%%%%%%%%%%%%%%%%%%%%%%%%%%%%%%%%%%%
\usepackage[dvips,
bookmarks,
bookmarksopen,
backref,
colorlinks,linkcolor={blue},citecolor={blue},urlcolor={blue},
]{hyperref}


%  This is a template for Papers to the Insight Journal. 
%  It is comparable to a technical report format.

% The title should be descriptive enough for people to be able to find
% the relevant document. 
\title{Boolean Operations for Surfaces in VTK Without External Libraries}

% 
% NOTE: This is the last number of the "handle" URL that 
% The Insight Journal assigns to your paper as part of the
% submission process. Please replace the number "1338" with
% the actual handle number that you get assigned.
%
\newcommand{\IJhandlerIDnumber}{1338}

% Increment the release number whenever significant changes are made.
% The author and/or editor can define 'significant' however they like.
\release{1.00}

% At minimum, give your name and an email address.  You can include a
% snail-mail address if you like.
\author{Cory Quammen and Russell M. Taylor II}
\authoraddress{Department of Computer Science\\
             The University of North Carolina at Chapel Hill
}

\begin{document}

%
% Add hyperlink to the web location and license of the paper.
% The argument of this command is the handler identifier given
% by the Insight Journal to this paper.
% 
\IJhandlefooter{\IJhandlerIDnumber}


\ifpdf
\else
   %
   % Commands for including Graphics when using latex
   % 
   \DeclareGraphicsExtensions{.eps,.jpg,.gif,.tiff,.bmp,.png}
   \DeclareGraphicsRule{.jpg}{eps}{.jpg.bb}{`convert #1 eps:-}
   \DeclareGraphicsRule{.gif}{eps}{.gif.bb}{`convert #1 eps:-}
   \DeclareGraphicsRule{.tiff}{eps}{.tiff.bb}{`convert #1 eps:-}
   \DeclareGraphicsRule{.bmp}{eps}{.bmp.bb}{`convert #1 eps:-}
   \DeclareGraphicsRule{.png}{eps}{.png.bb}{`convert #1 eps:-}
\fi


\maketitle


\ifhtml
\chapter*{Front Matter\label{front}}
\fi


% The abstract should be a paragraph or two long, and describe the
% scope of the document.
\begin{abstract}
\noindent
A previous VTK Journal article provided a VTK class that enabled boolean operations on surfaces defined by \code{vtkPolyData} objects. While the capabilities introduced by the previous contribution satisfied numerous request on the vtk-developers and vtk-users email lists, the class could not be incorporated into VTK because of a licensing incompatibility with the GTS library used to perform the boolean operations. This article describes a set of classes than enable the same boolean operations as the previous contribution using only VTK classes. Our contribution also has some feature enhancements over the previous contribution, namely,  support for meshes with more than one connected component and interpolation of point data at points introduced by the splitting procedure.

\end{abstract}

\IJhandlenote{\IJhandlerIDnumber}

\tableofcontents

\section{Clipping One Surface with Another Surface}

The most complex part of the computing boolean operations on a surface mesh is splitting one input surface mesh by the other. 

\subsection{Identifying Surface-Surface Intersection}

Identifying the intersection of a mesh with another mesh involves identifying the intersection between each triangle in the first mesh with the triangles that intersect it in the second mesh. A triangle-triangle intersection test yields two end points of the line segment defining the intersection for non-coplanar triangles; this contribution does not explicitly handle intersections between two coplanar triangles. 

Oriented bounding box (OBB) trees are used to accelerate the identification of triangle-triangle intersections between meshes. Two \code{vtkOBBTree} objects are instantiated, one for each input mesh. The method \code{vtkOBBTree::IntersectWithOBBTree()} is then called on one OBB tree with the second OBB tree and a callback function passed as parameters. The callback function performs exact triangle-triangle intersection tests between the triangles in overlapping nodes from the two OBB trees and stores the results in several data structures used later in the algorithm.

End points from the triangle-triangle intersections are stored in a \code{vtkPoints} object and the line cells are stored in a \code{vtkCellArray}. The points  are merged using a \code{vtkMergePoints} object, leading to many fewer connected components than if the line segments were stored as ``line soup''. In addition, each end point is tested to see if it lies on each edge of the two input triangles. If so, the index of the triangle, the index of the edge, and the index of the intersecting line is associated with the index of the endpoint in a \code{std::multimap}.

The end point-edge intersection information is used to split the fully connected intersection lines to respect the topology of each input mesh. Because the end points of the intersection lines are merged during construction, they need to be split where the mesh is split. The end result should be that a copy of each line end point is produced for each set of triangles that share an edge. This can be achieved by iterating over the elements in the end point-edge intersection \code{std::multimap}, creating a copy of the point, and removing all records in the multimap where the cell ID is the same as one identified as being a neighbor of the current cell being processed connected at the edge in question.

\subsection{Splitting the Mesh}

After the intersection lines are split for a mesh, the next step is to identify candidate cells for splitting. A cell is a candidate cell in two cases:

\begin{enumerate}
\item The cell index is in the intersection line map, meaning that intersection lines lie inside the cell.
\item The cell is the neighbor of a cell identified by case 1.
\end{enumerate}

The second case is important because one of the mesh-mesh intersection lines may have an endpoint on the edge of one cell, but no line that uses that endpoint on the cell neighbor across that edge. The cell neighbor needs to be split to avoid introducing a T-junction and therefore a hole in the output mesh.

Splitting proceeds on a cell-by-cell basis. For each cell that needs splitting, the following data is gathered:

\begin{enumerate}
\item The three points that define the cell.
\item The boundary lines of the cell.
\item The split intersection lines that lie in the cell.
\item The points that define the lines in 3.
\item Points from neighboring cells that lie on a cell edge but which were not already added in 4.
\end{enumerate}

The lines collected above are used to constrain the new triangulation of the points. That is to say, these lines should be present in the output mesh. Special processing of the boundary lines is required to achieve the desired split. Specifically, boundary lines must be split at points that lie on them.

The \code{vtkDelaunay2D} class can be used to achieve a constrained triangulation. A transformation of the points collected above that rotates them to the XY-plane is calculated is computed and set as the transform for the \code{vtkDelaunay2D} object. A new \code{vtkPolyData} object that contains the collected points and lines is defined and passed as both the input and constraint source to a \code{vtkDelaunay2D} object. The line point indices are renumbered to point to their locations in the new \code{vtkPolyData} object, and a map back to their original indices is created to remap the output cell indices from the triangulation to the point indices in the output mesh.

\section{Mesh-to-Mesh Signed Distance Calculation}

\section{Boolean Operations}

\subsection{Union}

\subsection{Intersection}

\subsection{Difference}

\section{Limitations}

These classes do not properly handle intersections between coplanar triangles. When such an intersection event has been detected, the intersection information is discarded. Such intersections likely occur when portions of the meshes are immediately adjacent to each other, however, and their exclusion from the surface-to-surface intersection information will usually not affect the mesh splitting results.

\section{Software Requirements}

You need to have the following software installed:

% The {itemize} environment uses a bullet for each \item.  If you want the 
% \item's numbered, use the {enumerate} environment instead.
\begin{itemize}
  \item  Insight Toolkit 2.4.
  \item  CMake 2.2
\end{itemize}

Note that other versions of the Insight Toolkit are also available in the
testing framework of the Insight Journal. Please refere to the following page
for details

\url{http://www.insightsoftwareconsortium.org/wiki/index.php/IJ-Testing-Environment}


% The preceding sections will have been written in a gentler,
% introductory style.  You may also wish to include a reference
% section, documenting all the functions/exceptions/constants.
% Often, these will be placed in separate files and input like this:



\appendix

\section{This is an Appendix}

To create an appendix in a Insight HOWTO document, use markup like
this:

\begin{verbatim}
\appendix

\section{This is an Appendix}

To create an appendix in a Insight HOWTO document, ....


\section{This is another}

Just add another \section{}, but don't say \appendix again.
\end{verbatim}


%%%%%%%%%%%%%%%%%%%%%%%%%%%%%%%%%%%%%%%%%%%%%%%%%%%%%%%%%%
%
%  Example on how to insert a figure
%
%%%%%%%%%%%%%%%%%%%%%%%%%%%%%%%%%%%%%%%%%%%%%%%%%%%%%%%%%%

\begin{figure}
\center
%\includegraphics[width=0.8\textwidth]{RegistrationComponentsDiagram.eps}
\itkcaption[Registration Framework Components]{The basic components of the
registration framework are two input images, a transform, a metric, an
interpolator and an optimizer.}
\label{fig:RegistrationComponents}
\end{figure}



%%%%%%%%%%%%%%%%%%%%%%%%%%%%%%%%%%%%%%%%%
%
%  Insert the bibliography using BibTeX
%
%%%%%%%%%%%%%%%%%%%%%%%%%%%%%%%%%%%%%%%%%

\bibliographystyle{plain}
\bibliography{Article}


\end{document}

